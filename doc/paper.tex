% This is samplepaper.tex, a sample chapter demonstrating the
% LLNCS macro package for Springer Computer Science proceedings;
% Version 2.20 of 2017/10/04
%
\documentclass[runningheads]{llncs}
%
\usepackage{graphicx}
\usepackage[utf8]{inputenc}
% Used for displaying a sample figure. If possible, figure files should
% be included in EPS format.
%
% If you use the hyperref package, please uncomment the following line
% to display URLs in blue roman font according to Springer's eBook style:
% \renewcommand\UrlFont{\color{blue}\rmfamily}

\begin{document}
%
\title{Versinus: an Animated Visualization Method for Evolving Networks\thanks{Supported by FAPESP}}
%
%\titlerunning{Abbreviated paper title}
% If the paper title is too long for the running head, you can set
% an abbreviated paper title here
%
\author{Renato Fabbri\inst{1}\orcidID{0000-0002-9699-629X} \and
Maria Cristina Ferreira de Oliveira\inst{1}\orcidID{0000-0002-4729-5104}}
%
\authorrunning{R. Fabbri and M. C. F. de Oliveira}
% First names are abbreviated in the running head.
% If there are more than two authors, 'et al.' is used.
%
\institute{University of São Paulo, São Carlos SP, BR\\
\email{renato.fabbri@gmail.com},
\email{cristina@icmc.usp.br}\\
\url{http://conteudo.icmc.usp.br/pessoas/cristina/}}
%
\maketitle              % typeset the header of the contribution
%
\begin{abstract}
% The abstract should briefly summarize the contents of the paper in
% 150--250 words.
Most real world networks change over time with new nodes and links or their removal, or even with metadata modification.
Increasing interest in modeling evolving networks has been reported and
a number of methods have been proposed to visualize them.  
  This article presents a novel visualization approach for (time-)evolving (or longitudinal) networks,
  specially useful for the observation of (in)stability of basic topological characteristics.
  The method has received a few software implementations and has been useful in research and publication, which motivated the writing of this present document.
  Named Versinus, the visualization consists essentially in placing the most connected nodes (hubs and intermediary) along a sine curve, and the peripheral nodes nodes along a separate segment, usually a line.
  Thus its name from  Latin \emph{versus} (line) and \emph{sinus} (sine).
  The method has provided insights into network properties and for network visualizations, which are described here after a formal presentation of the visualization and yielded software.


\keywords{Network visualization  \and Evolving networks \and Animated visualization \and Complex networks \and Data visualization.}
\end{abstract}
%
%
%
\section{Introduction}
The visualization of evolving networks poses challenges related to data size and complexity which have merited various contributions~\cite{evo1,evo2,evo3}.
\subsection{Related work}
% on animated visualization
% on vis of evo nets
Although animation is often criticized in the data visualization canonical literature~\cite{munzner,ware}, advantages have also been reported~\cite{cog,anim} e.g. making visuallizations more eye catching and due to the fact that we are accustumed to moving through the world and to moving objects.
Evolving network visualization has been tackled both through animated and static (often timeline-based) visualizations~\cite{ego,brain,visAn}, focusing on various global and local network features.
The contribution reported in this document is singular in consisting basically of a network layout which is fit for scale-free characteristics, together with some visualization artifacts dependent of the implementation.
The collection of procedures and conceptualizations involved in the achieving the final visualization were also considered worth documenting as were not found elsewhere in the literature, designed simple and genuine, and found of scientific relevance~\cite{stab}.

\subsection{Nomenclature}

\section{A description of the method}
\section{Software implementations}
\section{Insights gathered}
\section{Conclusions and further work}
%
% ---- Bibliography ----
%
% BibTeX users should specify bibliography style 'splncs04'.
% References will then be sorted and formatted in the correct style.
%
\bibliographystyle{splncs04}
\bibliography{paperbib}
%
% \begin{thebibliography}{8}
% \bibitem{ref_article1}
% Author, F.: Article title. Journal \textbf{2}(5), 99--110 (2016)
% 
% \bibitem{ref_lncs1}
% Author, F., Author, S.: Title of a proceedings paper. In: Editor,
% F., Editor, S. (eds.) CONFERENCE 2016, LNCS, vol. 9999, pp. 1--13.
% Springer, Heidelberg (2016). \doi{10.10007/1234567890}
% 
% \bibitem{ref_book1}
% Author, F., Author, S., Author, T.: Book title. 2nd edn. Publisher,
% Location (1999)
% 
% \bibitem{ref_proc1}
% Author, A.-B.: Contribution title. In: 9th International Proceedings
% on Proceedings, pp. 1--2. Publisher, Location (2010)
% 
% \bibitem{ref_url1}
% LNCS Homepage, \url{http://www.springer.com/lncs}. Last accessed 4
% Oct 2017
% \end{thebibliography}
\end{document}
